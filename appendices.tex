%% -*- coding:utf-8 -*-
\begin{appendices}

\chapter{Abstract algebra}

\section{Groups}

\begin{definition}[Group]
  \label{def:group}
  Let we have a set of elements $G$ with a defined binary operation
  $\circ$ that satisfied the following properties.
  \begin{enumerate}
  \item Closure: $\forall a, b \in G$: $a \circ b \in G$
  \item Associativity: $\forall a, b, c \in G$:
    $a \circ \left( b \circ c \right) =
    \left( a \circ b \right) \circ c$
  \item Identity element: $\exists e \in G$ such that
    $\forall a \in G$: $e \circ a = a \circ e = a$
  \item Inverse element: $\forall a \in G$ $\exists a^{-1} \in G$ such that
    $a \circ a^{-1} = e$
  \end{enumerate}
  In this case $\left(G, \circ\right)$ is called as group.
\end{definition}
Therefore the group is a \mynameref{def:monoid} with inverse element
property. 

\begin{example}[Group $\mathbb{Z}/2\mathbb{Z}$]
  Consider a set of 2 elements: $G = \left\{0, 1\right\}$ with the
  operation $\circ$ defined by the table \ref{tab:CayleyZ2Z}.
  \begin{table}
    \centering
    \caption{Cayley table for $\mathbb{Z}/2\mathbb{Z}$}
    \label{tab:CayleyZ2Z}
    \begin{tabular}{l|ll}
      \toprule
      $\circ$ & 0 & 1 \\
      \midrule
      0 & 0 & 1 \\
      1 & 1 & 0 \\
      \bottomrule
    \end{tabular}
  \end{table}

  The identity element is $0$ i.e. $e = 0$.
  Inverse element is the element itself
  because $\forall a \in G$: $a \circ a = 0 = e$.

  See also example \ref{ex:quotientgroup}
  \label{ex:groupZ2}
\end{example}

\begin{definition}[Abelian group]
  Let we have a \mynameref{def:group} $\left(G, \circ\right)$.
  The group is called an Abelian or commutative if
  $\forall a, b \in G$ it holds $a \circ b = b \circ a$.
  \label{def:abeliangroup}
\end{definition}


\section{Rings and Fields}

\subsection{Rings}

\begin{definition}[Ring]
  Consider a set $R$ with 2 binary operations defined. The first one
  $\oplus$ (addition) and elements of $R$ forms an
  \mynameref{def:abeliangroup}
  under this operation. The second one is $\odot$ (multiplication) and
  the elements of $R$ forms a \mynameref{def:monoid} under 
  the operation. The two binary operations are connected each other
  via the following distributive law
  \begin{itemize}
  \item Left distributivity:
    $\forall a,b,c \in R$:
    $a \odot \left(b \oplus c\right) =
    a \odot b \oplus a \odot c$
  \item Right distributivity:
    $\forall a,b,c \in R$:
    $\left( a \oplus b \right) \odot c =
    a \odot c \oplus b \odot c$
    
  The identity element for $\left(R, \oplus\right)$ is denoted as $0$
  (additive identity).
  The identity element for $\left(R, \odot\right)$ is denoted as $1$
  (multiplicative identity).

  The inverse element to $a$ in $\left(R, \oplus\right)$ is denoted as $-a$
  \end{itemize}

  In this case $\left(R, \oplus, \odot\right)$ is called as ring.
  \label{def:ring}
\end{definition}

The \mynameref{def:ring} is a generalization of integer numbers conception.
\begin{example}[Ring of integers $\mathbb{Z}$]
  The set of integer numbers $\mathbb{Z}$ forms a \mynameref{def:ring}
  under $+$ and $\cdot$ operations i.e. addition $\oplus$ is
  $+$ and multiplication $\odot$ is $\cdot$. Thus for integer
  numbers we have the following \mynameref{def:ring}:
  $\left(\mathbb{Z}, +, \cdot\right)$
  \label{ex:ring}
\end{example}

\subsection{Fields}

\begin{definition}[Field]
  The ring $\left(R, \oplus, \odot\right)$ is called as a field if
  $\left(R \setminus \{0\}, \odot\right)$ is an \mynameref{def:abeliangroup}.

  The inverse element to $a$ in
  $\left(R \setminus\{0\}, \odot\right)$ is denoted as $a^{-1}$
  \label{def:field}
\end{definition}

\begin{example}[Field $\mathbb{Q}$]
  Note that $\mathbb{Z}$ is not a field because not for every integer
  number an inverse exists. But if we consider a set of fractions
  $\mathbb{Q} = \left\{a/b \mid a \in \mathbb{Z}, b \in
  \mathbb{Z}\setminus\{0\}\right\}$ when it will be a field.

  The
  inverse element to $a/b$  in
  $\left(\mathbb{Q}\setminus\{0\}, \cdot\right)$  will be $b/a$.
  \label{ex:field}
\end{example}

\end{appendices}

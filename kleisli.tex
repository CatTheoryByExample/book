%% -*- coding:utf-8 -*-
\chapter{Kleisli category}

\begin{definition}[Kleisli category]
\label{def:kleisli_category}
Let $\cat{C}$ is a category, $M$ is an \mynameref{def:endofunctor} and
$\left<M, \mu, \eta\right>$ is a \mynameref{def:monad}. Then we can
construct a new category $\cat{C_M}$ as follows:
\begin{eqnarray}
\catob{C_M} = \catob{C},
\nonumber \\
\hom_{\cat{C_M}}\left(a, b\right) = 
\hom_{\cat{C}}\left(a, M(b)\right)
\nonumber
\end{eqnarray}
i.e. objects of categories $\cat{C}$ and $\cat{C_M}$ are the same but
morphisms from $\cat{C_M}$ form a subset of morphisms $\cat{C}$:
$\cathom{C_M} \subset \cathom{C}$. The category is called as
\textit{Kleisli category}. 

The identity morphism in the Kleisli category is the
\mynameref{def:nt} $\eta$ \eqref{eq:monad_eta} defined by the monad
$\left<M, \mu, \eta\right>$: 
\[
\idm{C_M} = \eta
\]
\end{definition}

\begin{remark}[Kleisli category composition]
\mynameref{def:kleisli_category} has a non trivial composition rules.
If we have 2 \mynameref{def:morphism}s from $\cathom{C_M}$:
\[
f_M: a \to b
\]
and
\[
g_M: b \to c.
\]
The morphisms have correspondent ones in $\cat{C}$:
\[
f: a \to M(b)
\]
and
\[
g: b \to M(c).
\]
The composition $g_M \circ f_M$ gives a new morphism
\[
h_M = g_M \circ f_M: a \to c.
\]
The corresponding one from $\cat{C}$ is
\[
h: a \to M(c).
\]
It has to be pointed that the compositions in $\cat{C}$ and
$\cat{C_M}$ are not the same:
\[
g_M \circ f_M \ne g \circ f.
\]
\end{remark}

\mynameref{def:kleisli_category} widely spread in programming
especially it provides good description for different types of
computations, for instance \cite{bib:Moggi91, bib:milewski2018category} 
\begin{itemize}
\item \textbf{Partiality} i.e. then a function not defined for each input, for
  instance the following expression is undefined (or partially
  defined) for $x = 0$: $f(x) = \frac{1}{x}$
\item \textbf{Non-Determinism} i.e. then multiply output are possible
\item \textbf{Side-effects} i.e. then a function communicates with
  an environment
\item \textbf{Exception} i.e. when some input is incorrect and can
  produce an abnormal result. Therefore it is the same as
  \textbf{Partiality} and will be considered below as the same type of
  computation. 
\item \textbf{Continuation} i.e. when we need to save the current
  state of the computation and be able to restore it on demand later
\item \textbf{Interactive input} i.e. a function that reads data from
  an input device (keyboard, mouse, etc.)
\item \textbf{Interactive output} i.e. a function that writes data to
an output device (monitor etc.)
\end{itemize}

\section{Partiality and Exception}

Partial functions and exceptions can be processed via monad be called
as Maybe. There will be implementations in different languages below.
And the usage example for the following function implementation
\[
h(x) = \frac{1}{2 \sqrt{x}}.
\]
The function is a composition of 3 functions:
\begin{eqnarray}
f_1(x) = \sqrt{x},
\nonumber \\
f_2(x) = 2 \cdot x,
\nonumber \\
f_3(x) = \frac{1}{x}
\label{eq:monadmaybe_ex_f}
\end{eqnarray}
and as result the goal can be implemented as the following
composition:
\begin{equation}
h = f_3 \circ f_2 \circ f_1.
\label{eq:monadmaybe_ex_h}
\end{equation}
$f_2$ is a \mynameref{def:pure_function} and defined $\forall x \in \mathbb{R}$. The
functions $f_1, f_3$ are partially defined.

\subsection{Haskell example}

\begin{example}[Maybe monad][$\cat{Hask}$]
\label{ex:maybe_monad_haskell}
The Maybe monad can be implemented as follows
\begin{minted}{haskell}
  instance Monad Maybe where
    return = Just
    join Just( Just x) = Just x
    join _ = Nothing
\end{minted} 

Our functions \eqref{eq:monadmaybe_ex_f} can be implemented as follows
\begin{minted}{haskell}
f1 :: (Ord a, Floating a) => a -> Maybe a
f1 x = if x >= 0 then Just(sqrt x) else Nothing 

f2 :: Num a => a -> Maybe a
f2 x = Just (2*x)

f3 :: (Eq a, Fractional a) => a -> Maybe a
f3 x = if x /= 0 then Just(1/x) else Nothing
\end{minted}

The $h$ \eqref{eq:monadmaybe_ex_h} is the composition via bind
operator:
\begin{minted}{haskell}
h :: (Ord a, Floating a) => a -> Maybe a
h x = (return x) >>= f1 >>= f2 >>= f3
\end{minted}

The usage example is the following:
\begin{minted}{bash}
*Main> h 4
Just 0.25
*Main> h 1
Just 0.5
*Main> h 0
Nothing
*Main> h (-1)
Nothing
\end{minted}

\end{example}

\subsection{C++ example}

\begin{example}[Maybe monad][\textbf{C++}]
\label{ex:maybe_monad_cpp}
The Maybe monad can be implemented as follows
\begin{minted}{c++}
template <class A> using Maybe = std::optional<A>;

template < class A, class B> 
Maybe<B> fmap(std::function<B(A)> f, Maybe<A> a) {
  if (a) {
    return f(a.value());
  }
  return {};
}

template < class A> 
Maybe<A> pure(A a) {
  return a;
}

template < class A> 
Maybe<A> join(Maybe< Maybe<A> > a){
  if (a) {
    return a.value();
  }
  return {};
}
\end{minted} 

Our functions \eqref{eq:monadmaybe_ex_f} can be implemented as follows
\begin{minted}{c++}
std::function<Maybe<float>(float)> f1 =
    [](float x) {
      if (x >= 0) {
        return Maybe<float>(sqrt(x));
      }
      return Maybe<float>();
    };

std::function<Maybe<float>(float)> f2 = [](float x) { return 2 * x; };

std::function<Maybe<float>(float)> f3 =
    [](float x) {
      if (x != 0) {
        return Maybe<float>(1 / x);
      }
      return Maybe<float>();
    };
}
\end{minted}

The $h$ \eqref{eq:monadmaybe_ex_h} is the composition via bind
operator:
\begin{minted}{c++}
auto h(float x) {
  Maybe<float> a = pure(x);
  return bind(f3,bind(f2,bind(f1, a)));
};
\end{minted}

\end{example}

\section{Non-Determinism}

The situation when a function returns several values is not applicable
for \mynameref{def:setcategory} but can appear for
\mynameref{def:relcategory}. From other hand the non standard
situation is required for practical applications and as result has to
be modeled in programming languages. The \textbf{List} monad is used
for it.

\subsection{Haskell example}

\begin{example}[List monad][$\cat{Hask}$]
\label{ex:list_monad_haskell}
\begin{minted}{haskell}
  instance Monad [] where
    return x = [x]
    join = concat
\end{minted}
\end{example}

\section{Side effects and interactive input/output}

TBD

\section{Continuation}

TBD


%% -*- coding:utf-8 -*-
\chapter*{Introduction}

You just looked at yet another introduction to Category Theory. The
subject mostly consists of a lot of definitions that are related each
others and we wrote the book to collect all of them in one
place to be easy checked and updated in future when we decide to refresh
our knowledge about the field of math. Therefore the book was written mostly
for our category theory studying purposes but we will appreciate if
somebody else find it useful. 

The topics(chapters) cover the base definitions
(\mynameref{def:object}, \mynameref{def:morphism} and
\mynameref{def:category}), \mynameref{def:functor},
\mynameref{def:nt}, \mynameref{def:monad} and also include important
results from the category theory such as Yoneda's lemma (see
\cref{sec:yoneda}) and Curry-Howard-Lambek correspondence (see
\cref{sec:curry_howard_lambek}). The \cref{sec:topos} gives an
introduction to the topos theory i.e. just another view of the
\mynameref{def:set}s.

There are a lot of examples in each chapter. The examples cover
different category 
theory application areas. We assume that the reader is familiar with
the corresponding area and the example(s) can be passed if not. I.e.
anyone can choose the suitable example(s) for (s)he. 

The most important examples are related to the set theory. The set
theory and category theory are very close related. Each one can be
considered as an alternative view to another one.

There are also a lot of examples from programming languages which include
Haskell, Scala, C\texttt{++}. The source files for programming languages 
examples (Haskell, C\texttt{++}, Scala) can be found on github repositories:
\begin{itemize}
\item Haskell: \cite{github:cattheory_hs_examples}
\item Scala: \cite{github:cattheory_scala_examples}
\item C\texttt{++}: \cite{github:cattheory_cpp_examples}
\end{itemize}

The examples from physics are related to quantum mechanics that is the
most known for me. For the examples We were inspired by the Bob Coecke
article \cite{bib:arxiv:Bob_Coecke_2008}.

There is also additional material related to abstract algebra (see
\cref{sec:abstractalgebra}) taken from
\cite{github:galois_ivanmurashko}. The material describes the
different math constructions used in the book.

The text is distributed under \textbf{Creative Common Public License}
(see the text of the license at the end of the book)
i.e. any reader has right to copy, store, modify, distribute or build
upon the book until the original authors are pointed in the derivative
products. The initial text of the book can be found at
\cite{github:cattheory_ivanmurashko}.  

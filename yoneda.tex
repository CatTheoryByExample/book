%% -*- coding:utf-8 -*-
\chapter{Yoneda's lemma}
\label{sec:yoneda}
We will assume that the category $\cat{C}$ to be a
\mynameref{def:localy_small_category}.
\section{Hom functor}

\begin{definition}[Covariant Hom functor]
\label{def:cov_hom_functor}
Let $\cat{C}$ is a \mynameref{def:localy_small_category} and $a \in
\catob{C}$. Consider \mynameref{def:functor} from $\cat{C}$ to the
\mynameref{def:setcategory} defined by the following rules
\begin{itemize}
\item $\forall x \in \catob{C}$ define an object in the set category:
  $Hom_c(a, x) \in \catob{Set}$ 
\item $\forall f: x \to y \in \cathom{C}$ define a function in the set category
  $Hom_c(a, f): Hom_c(a, x) \to Hom_c(a, y)$.
\end{itemize}  
The \textit{covariant Hom functor} is denoted as $Hom_c(a, -)$.
\end{definition}

\begin{definition}[Contravariant Hom functor]
\label{def:con_hom_functor}
Let $\cat{C}$ is a \mynameref{def:localy_small_category}
TBD

The \textit{contravariant Hom functor} is denoted as $Hom_c(-, a)$.
\end{definition}


TBD

\section{Yoneda's lemma}

TBD

\section{Examples}

\subsection{Quantum mechanics}

\subsubsection{Flori interpretation of quantum mechanics}
TBD

%% -*- coding:utf-8 -*-
\chapter{Functors}

\section{Definitions}

\begin{definition}[Functor]
  \label{def:functor}
  Let $\cat{C}$ and $\cat{D}$ are 2 categories. A mapping $F: \cat{C}
  \tof \cat{D}$ between the categories is called \textit{functor} if it
  preserves the internal structure (see \cref{fig:functor}): 
  \begin{itemize}
  \item $\forall a_C \in \catob{C}, \exists a_D \in \catob{D}$ such that
    $a_D = F( a_C )$
  \item $\forall f_C \in \cathom{C}, \exists f_D \in \cathom{D}$ such
    that $\dom f_D = F (\dom f_C), \cod f_D = F (\cod f_C)$. We will use
    the following notation later: $f_D = F(f_C)$.
  \item $\forall f_C, g_C$ the following equation holds: 
    \[
    F\left(f_C \circ
    g_C\right) = F\left(f_C\right) \circ F\left(g_C\right) = f_D \circ
    g_D.
    \]
  \item $\forall x \in \catob{C}: F(\idm{x}) = \idm{F(x)}$.
  \end{itemize}  

  \begin{figure}
    \centering
    \begin{tikzpicture}[ele/.style={fill=black,circle,minimum
          width=.8pt,inner sep=1pt},every fit/.style={ellipse,draw,inner
          sep=-2pt}]

      % the texts

      \node at (0,3) {$C$};        
      \node at (4,3) {$D$};        

      \node[ele,label=above:$a_C$] (ac) at (0,2) {};    
      \node[ele,label=below:$b_C$] (bc) at (0,0) {};    
      \node[ele,label=above:$a_D$] (ad) at (4,2) {};
      \node[ele,label=below:$b_D$] (bd) at (4,0) {};

      \node[draw,fit= (ac) (bc),minimum width=2cm, minimum
        height=3.5cm] {} ;
      \node[draw,fit= (ad) (bd),minimum width=2cm, minimum
        height=3.5cm] {} ;

      \draw[->,thick,shorten <=2pt,shorten >=2pt] (ac) to
      node[left]{$f_C$} (bc);
      \draw[->,thick,shorten <=2pt,shorten >=2pt] (ad) to
      node[right]{$f_D$} (bd);
      \draw[->,thick,shorten <=2pt,shorten >=2pt] (ac) to
      node[sloped,above]{$a_D = F(a_C)$} (ad);
      \draw[->,thick,shorten <=2pt,shorten >=2pt] (bc) to
      node[sloped,above]{$b_D = F(b_C)$} (bd);
    \end{tikzpicture}
    \caption{Functor $F: \cat{C} \tof \cat{D}$ definition}
    \label{fig:functor}
  \end{figure}
\end{definition}

\begin{remark}[Functor]
  When we say that functor preserve internal structure we assume that
  the functor is not just mapping between \mynameref{def:object}s but
  also between \mynameref{def:morphism}s.  

  Thus functor is something that allows map one category into another.
  The initial category can be considered as a pattern thus the mapping
  is some kind of searching of the pattern inside another category.
\end{remark}

Programming languages can be considered as a good platform for the
functor examples. 
The functor can be defined in Haskell as follows
\footnote{the real definition is quite different from the current one}
\begin{example}[Functor][$\cat{Hask}$]
  \label{ex:functor_haskell}
  \begin{minted}{haskell}
    class Functor f where
    fmap :: (a -> b) -> f a -> f b
  \end{minted} 
\end{example}

In Scala it can be defined in the same way
\begin{example}[Functor][$\cat{Scala}$]
  \label{ex:functor_scala}
  \begin{minted}{scala}
    trait Functor[F[_]] {
      def fmap[A, B](f: A => B): F[A] => F[B]
  }\end{minted} 
\end{example}

In C++ the definition differs
\begin{example}[Functor][$\cat{C++}$]
  \label{ex:functor_cpp}
  In C++ templates can be considered as type constructors in Haskell
  and therefore can convert one type for another. For instance the
  list of strings can be got with the following construction:

  \begin{minted}{c++}
    using StringList = std::list<std::string>;
    StringList a = {"1", "2", "3"};
  \end{minted} 
  i.e. we have \mynameref{def:object}s mapping out of the box.
  Therefore we need to define fmap
  operation for \mynameref{def:morphism}s mapping to complete the
  \mynameref{def:functor} definition. It can be declared as 
  follows 
  \begin{minted}{c++}
    template < template< class ...> class F, class A, class B> 
    F<B> fmap(std::function<B(A)>, F<A>);
  \end{minted}
  The template specialization for the \textbf{std::list} can be
  written as follows
  \begin{minted}{c++}
    // file: functor.h
    template <class A, class B>
    std::list<B> fmap(std::function<B(A)> f, std::list<A> a) {
      std::list<B> res;
      std::transform(a.begin(), a.end(), back_inserter(res), f);
      return res;
    }
  \end{minted} 

  The simple usage example is the following
  \begin{minted}{c++}
    StringList a = {"1", "2", "3"};
    std::function<int(std::string)> f = [](std::string s) {
      return 2 * atoi(s.c_str());
    };
    auto res = fmap<>(f, a);
  \end{minted}
\end{example}


\begin{definition}[Endofunctor]
  \label{def:endofunctor}
  Let $\cat{C}$ is a \mynameref{def:category}. The
  \mynameref{def:functor} $E: \cat{C} \tof \cat{C}$ i.e. 
  the functor from a category to the same category is called
  \textit{endofunctor}. 
\end{definition}

\begin{definition}[Identity functor]
  \label{def:idfunctor}
  Let $\cat{C}$ is a \mynameref{def:category}. The
  \mynameref{def:functor} $\idf{C}: \cat{C} \tof \cat{C}$ is called \textit{identity
    functor} if for every object $a \in \catob{C}$
  \[
  \idf{C}(a) = a
  \]
  and for every \mynameref{def:morphism} $f \in \cathom{C}$
  \[
  \idf{C}(f) = f
  \] 
\end{definition}

\begin{remark}[Identity functor]
  \label{rem:idfunctor}
  First of all notice that \mynameref{def:idfunctor} is an
  \mynameref{def:endofunctor}.

  There is difference between identity functor and \mynameref{def:id}
  because the first one has deal with both \mynameref{def:object}s and
  \mynameref{def:morphism}s while the second one with the objects
  only. 
\end{remark}

\begin{definition}[Functor composition]
  \label{def:functor_composition}
  If we have 3 categories $\cat{C}, \cat{D}, \cat{E}$ and 2 functors
  between them: $F: \cat{C} \tof \cat{D}$ and $G: \cat{D} \tof \cat{E}$
  then we can construct a new functor $H: \cat{C} \tof \cat{E}$ that is
  called \textit{functor composition} and denoted as $H = G \circ F$.
  TBD
\end{definition}

\section{\textbf{Cat} category}
The \mynameref{def:functor_composition} is associative by definition.
Therefore \mynameref{def:idfunctor} with the associative composition
allow us to define a category where other categories are considered as
objects and functors as morphisms: 
\begin{definition}[$\cat{Cat}$ category]
  \label{def:catcategory}
  The category of small categories (see \mynameref{def:small_category})
  denoted as $\cat{Cat}$ is the \mynameref{def:category} where objects
  are small categories and morphisms are \mynameref{def:functor}s
  between them.
\end{definition}

We can construct an extension of Cartesian product as follows
\begin{definition}[Category Product]
  \label{def:category_product}
  If we have 2 categories $\cat{C}$ and $\cat{D}$ then we can construct
  a new category $\cat{C} \times \cat{D}$ with the following components:
  \begin{itemize}
  \item \mynameref{def:object}s are the pairs $(c,d)$ where $c \in
    \catob{C}$ and $d \in \catob{D}$
  \item \mynameref{def:morphism}s are the pair $(f,g)$ where $f \in
    \cathom{C}$ and $g \in \cathom{D}$
  \item \mynameref{axm:composition} is defined as follows 
    \(
    (f_1, g_1) \circ (f_2, g_2) = (f_1 \circ f_2, g_1 \circ g_2)
    \)
  \item Identity is defined as follows: $\idm{C \times D} = 
    \left(\idm{C}, \idm{D}\right)$
  \end{itemize}
\end{definition}

\begin{definition}[Constant functor]
  \label{def:const_functor}
  Let consider a trivial functor $\Delta_c$ from \mynameref{def:category}
  $\cat{A}$ to category $\cat{C}$ such that $\forall a \in \catob{A}:
  \Delta_c a = c$ -fixed object in $\cat{C}$ and 
  $\forall f \in \cathom{A}: \Delta_c f = \idm{c}$. The trivial
  functor is called \textit{constant functor}.
\end{definition}

\begin{example}[Initial object][$\cat{Cat}$]
  \label{ex:initial_object_cat}
  \mynameref{def:empty_category} is the \mynameref{def:initial_object}
  in $\cat{Cat}$ category \cite{bib:stackexchange:empty_category}.
\end{example}

\begin{example}[Terminal object][$\cat{Cat}$]
  \label{ex:terminal_object_cat}
  \mynameref{def:trivial_category} is the \mynameref{def:terminal_object}
  in $\cat{Cat}$ category.
\end{example}

\begin{example}[Terminal object][$\cat{Fun}$]
  \label{ex:terminal_object_cat}
  \mynameref{def:const_functor} is the \mynameref{def:terminal_object}
  in $\cat{Fun}$ category.
\end{example}

The good example can be found in $\cat{Hask}$ category.
\begin{example}[Constant functor][$\cat{Hask}$]
  \label{ex:const_functor_hask}
  \begin{minted}{haskell}
    data Const c a = Const c
    fmap :: (a -> b) -> Const c a -> Const c b
    fmap f (Const c a) = Const c
  \end{minted}
\end{example}

\section{Contravariant functor}
\begin{definition}[Contravariant functor]
  \label{def:contravariant_functor}
  If we have categories $\cat{C}$ and $\cat{D}$ then the
  \mynameref{def:functor} $\cat{C^{op}} \tof \cat{D}$ is called
  \textit{contravariant functor}. 
\end{definition}

\begin{example}[Contravariant functor][$\cat{Hask}$]
  \label{ex:contravariant_functor_hask}
  Function mapping inside a functor is made via \textbf{fmap} (see
  \cref{ex:functor_haskell}) but sometimes the function that has to be
  mapped is \textbf{a -> b} but the result mapping has an inverse
  order: \textbf{f b -> f a}. In the case the contravariant functor
  can help
  \begin{minted}{haskell}
    class Contravariant f where
    contramap :: (a -> b) -> f b -> f a
  \end{minted}

  The contravariant functor should follow the following laws
  \begin{minted}{haskell}
    contramap id = id
    contramap f . contramap g = contramap (g . f)
  \end{minted}

  Consider the following task. We have a predicate for \textbf{Int}
  type that returns \textbf{True} if the number is greater than
  \textbf{10} otherwise it returns \textbf{False}:
  \begin{minted}{haskell}
    newtype Predicate a = Predicate { runPredicate :: a -> Bool}

    intgt10 :: Predicate Int
    intgt10 = Predicate ( \i -> i > 10 )
  \end{minted}
  Now we want to create a predicate that accepts a string and
  verify it length greater than 10 or not.
  I.e. we want to have something of the following type:
  \begin{minted}{haskell}
    strgt10 :: Predicate String
  \end{minted}
  In the case the \mynameref{def:contravariant_functor} helps.

  \begin{minted}{haskell}
    instance Contravariant Predicate where
    contramap f (Predicate p) = Predicate ( p . f )

    strgt10 :: Predicate [Char]
    strgt10 = contramap length intgt10
  \end{minted}
\end{example}


\section{Bifunctors}

\begin{definition}[Bifunctor]
  \label{def:bifunctor}
  Bifunctor is a \mynameref{def:functor} whose \mynameref{def:domain} is
  a \mynameref{def:category_product}. I.e. if $\cat{C_1}, \cat{C_2},
  \cat{D}$ are 3 categories then the \mynameref{def:functor} 
  \(
  F: \cat{C_1} \times \cat{C_2} \tof \cat{D}
  \) is called \textit{bifunctor}.
\end{definition}

\begin{example}[Bifunctor][$\cat{Set}$]
  \label{ex:product_bifunctor}
  Lets $A,B,C$ and $D$ are sets and $f: A \to C, g: B \to D$ are two
  \mynameref{def:function}s. Then the \mynameref{def:cartesian_product}
  with \mynameref{def:product_of_morphisms} form a
  \mynameref{def:bifunctor} $\times$.
\end{example}

\begin{example}[Maybe as a bifunctor][$\cat{Hask}$]
  \label{ex:maybe_functor}
  Lets show how the \textbf{Maybe a} type can be
  constructed from different 
  \mynameref{def:functor}s and as result show that the
  \textbf{Maybe a} is also a
  \mynameref{def:functor}. 
  \begin{minted}{haskell}
    data Maybe a = Nothing | Just a
    -- This is equivalent to
    data Maybe a = Either () (Identity a)
    -- Either is a bifunctor and () == Const () a 
    -- Thus Maybe is a composition of 2 functors 
  \end{minted}
\end{example}

\begin{definition}[Profunctor]
  \label{def:profunctor}
  If we have a category $\cat{C}$ then the \mynameref{def:bifunctor}
  $\cat{C^{op}} \times \cat{C} \tof \cat{C}$ is called
  \textit{profunctor}. 
\end{definition}

\begin{example}[Profunctor][$\cat{Hask}$]
  \label{ex:contravariant_functor_hask}
  TBD
  \begin{minted}{haskell}
    class Profunctor p where
    dimap :: (a' -> a) -> ( b -> b' ) -> p a b -> p a' b'
    -- p a b == a -> b
    dimap f g h = g . h . f
  \end{minted}
\end{example}





